%
% macros definies dans "Le Petit Livre De TeX de Raymond Seroul"
%

\catcode`\@=11 	% le temps de tout definir, c'est un caractere.

%
% note de bas de page (Ch. 6 : page 85)
% \note{texte}

\newcount\notenumber \notenumber=1
\def\note#1{\footnote{$^{\the\notenumber}$}{#1}%
\global\advance\notenumber by 1}

% 
% centrer un titre (Index : page 238)
% 

\long\def\ctitre#1{\vbox{\leftskip=0pt plus 1fil \rightskip=0pt plus 1fil \parfillskip=0pt #1}}

%
% pour faire un sommaire
%

\def\page#1{\leaders\hbox to 5mm{\hfil.\hfil}\hfill\rlap{\hbox to 10mm{\hfill#1}}\par}

% 
% pour les tableaux (Ch. 9 : Les tableaux)
% 

% traits verticaux invisibles et visibles
\def\tvi{\vrule height 12pt depth 5pt width 0pt}
\def\Tvi#1{ 	\ifnum #1=0	\vrule height  4pt depth 1pt width 0pt  \fi
	 	\ifnum #1=1	\vrule height 10pt depth 3pt width 0pt  \fi
	   	\ifnum #1=2	\vrule height 11pt depth 4pt width 0pt  \fi
	   	\ifnum #1=3	\vrule height 12pt depth 5pt width 0pt  \fi
	   	\ifnum #1=4	\vrule height 13pt depth 5pt width 0pt  \fi
	   	\ifnum #1=5	\vrule height 14pt depth 6pt width 0pt  \fi
	   	\ifnum #1=6	\vrule height 48pt depth 20pt width 0pt  \fi
	   }	
\def\tv{\tvi\vrule}
\def\Tv#1{\Tvi#1\vrule}

\def\hfq{\hfill\quad}  
\def\qhf{\quad\hfill}  
%pour centrer avec un quadratin a dte et a gche
\def\cc#1{\hfill\quad#1\quad\hfill}  
%pour centrer avec un demi quadratin a dte et a gche
\def\dd#1{\hfill\enspace#1\enspace\hfill}
%pour centrer avec un espace a dte et a gche
\def\ee#1{\hfill\ #1\ \hfill} 
%pour centrer sans rien  a dte ni a gche
\def\ff#1{\hfill #1\hfill} 

%%
%% guillemets francais
%%
%\def\og{\leavevmode\raise.3ex\hbox{$\scriptscriptstyle\langle\!\langle\,$}}
%\def\osg{\leavevmode\raise.3ex\hbox{$\scriptscriptstyle\langle\,$}}
%\def\fg{\leavevmode\raise.3ex\hbox{$\scriptscriptstyle\,\rangle\!\rangle$}}
%\def\fsg{\leavevmode\raise.3ex\hbox{$\scriptscriptstyle\,\rangle$}}
\def\ogui{\leavevmode\raise.3ex\hbox{$\scriptscriptstyle\langle\!\langle\,$}}
\def\fgui{\leavevmode\raise.3ex\hbox{$\scriptscriptstyle\,\rangle\!\rangle$}}

% numerotation romaine
\def\numrom#1{\uppercase\expandafter{\romannumeral #1 }}

% numerotation ancien style (utilise pour les annees, par exemple)
\let\an=\oldstylenums

% etoile en mode non mathematique
\def\stern{$\star$}

% pour les exposants et indices en mode non mathematique (chap 12 p217)

\def\up#1{\raise 1ex\hbox{\rmneu #1}}
\def\dw#1{\raise -0.5ex\hbox{\rmneu #1}}
\def\ind#1{\dw{\rmset #1}}

%
% lettrine
% #1 largeur indentee
% #2 nb de lignes a indenter
% #3 lettres a inserer
% \lettrine{30mm}{4}{L}'homme etc

\def\lettrine#1#2#3{\noindent\hangindent#1\hangafter-#2
		    \hskip-#1\smash{\hbox to#1{#3\hfill}}\ignorespaces}


%
% entourer une boite #2 d'un liseret d'epaisseur #1 (qq pt) page 123
% \boxit{qq pt}{la boite contenant le texte}
% \boxitxy permet de specifier un ecart different en x et en y
% \trait trace juste un trait a gauche de la boite 

\def\boxit#1#2{\setbox1=\hbox{\kern#1{#2}\kern#1}%
\dimen1=\ht1 \advance\dimen1 by #1 \dimen2=\dp1 \advance\dimen2 by #1
\setbox1=\hbox{\vrule height\dimen1 depth\dimen2\box1\vrule}%
\setbox1=\vbox{\hrule\box1\hrule}%
\advance\dimen1 by .4pt \ht1=\dimen1
\advance\dimen2 by .4pt \dp1=\dimen2 \box1\relax}

\def\boxitxy#1#2#3{\setbox1=\hbox{\kern#1{#3}\kern#1}%
\dimen1=\ht1 \advance\dimen1 by #2 \dimen2=\dp1 \advance\dimen2 by #2
\setbox1=\hbox{\vrule height\dimen1 depth\dimen2\box1\vrule}%
\setbox1=\vbox{\hrule\box1\hrule}%
\advance\dimen1 by .4pt \ht1=\dimen1
\advance\dimen2 by .4pt \dp1=\dimen2 \box1\relax}

\def\trait#1#2{\setbox1=\vbox{\kern#1{#2}\kern#1}%
\dimen1=\ht1 \advance\dimen1 by #1 \dimen2=\dp1 \advance\dimen2 by #1
\setbox1=\hbox{\vrule height\dimen1 depth\dimen2\box1}%
\box1\relax}

%
% pour ajouter des choses au dessus ou en dessous d'un symbole mathematique
%

\def\build#1#2^#3{\mathrel{\mathop{\kern 0pt#1}\limits_{#2}^{#3}}}

%
% systeme d'equations : page 192
%

\def\system#1{\left\{\null\,\vcenter{\openup1\jot\m@th
\ialign{\strut\hfil$##$&$##$\hfil&&\enspace$##$\enspace&
\hfil$##$&$##$\hfil\crcr#1\crcr}}\right.}

%
% diagram page 190 (autre version dans le ed2 p 160.)
%
\def\hfl#1#2{\smash{\mathop{\hbox to 12mm{\rightarrowfill}}\limits^{\scriptstyle #1}_{\scriptstyle #2}}}
\def\hlf#1#2{\smash{\mathop{\hbox to 12mm{\leftarrowfill}}\limits^{\scriptstyle #1}_{\scriptstyle #2}}}
\def\vfl#1#2{\llap{$\scriptstyle #1$}\left\downarrow\vbox to 6mm{}\right.\rlap{$\scriptstyle #2$}}
\def\vlf#1#2{\llap{$\scriptstyle #1$}\left\uparrow\vbox to 6mm{}\right.\rlap{$\scriptstyle #2$}}
\def\diagram#1{\def\normalbaselines{\baselineskip=0pt\lineskip=10pt\lineskiplimit=1pt} \matrix{#1}}

%
% Eqalign, version modifi\'ee de equalign
%
\def\Eqalign#1{\null\,\vcenter{\openup\jot\m@th\ialign{
\strut\hfil$\displaystyle{##}$&$\displaystyle{{}##}$\hfil&&\quad
\strut\hfil$\displaystyle{##}$&$\displaystyle{{}##}$\hfil\crcr#1\crcr}}\,}

%
% EEqalign. comme system, mais sans l'accolade.
%
\def\EEqalign#1{\left.\null\,\vcenter{\openup1\jot\m@th
\ialign{\strut\hfil$##$&$##$\hfil&&\enspace$##$\enspace&
\hfil$##$&$##$\hfil\crcr#1\crcr}}\right.}

%
% ecriture sur deux colonnes (TeXbook p417) 10 9bre 2
%
%  emploi :
%  \begindoublecolumns
%   texte
%  \enddoublecolumns
%
% Si on veut preciser soit meme la largeur des colonnes, il faut ecrire
% \begindoublecolumnsdim{N}, ou N est la largeur en mm que l'on desire donner a la colonne.
%
% Remarque : 
%  1) il y une autre solution pour ecrire avec deux colonnes : voir le
% TeXbook p257. cette solution permet aussi d'ecrire sur 3 colonnes
%  2) cette macro gere les notes de bas de page, mais les met en bas en colonne unique.
%     D. Knuth propose de faire � titre d'exercice la modification pour associer
%     a chaque colonne ses notes de bas de page

\newdimen\pagewidth \pagewidth=\hsize
\newbox\partialpage							% cree une boite privee nommee partialpage qui va
									% servir a faire tout le montage
\def\begindoublecolumns{\begingroup
  \output={\global\setbox\partialpage=\vbox{\unvbox255\bigskip}}\eject	% il met toute la page (box255) dans sa 
  \output={\doublecolumnout} \hsize=19truepc \vsize=89truepc}		%  boite privee, avec possibilite de la
				                                    	%  couper en morceaux (\unvbox)

\def\begindoublecolumnsdim#1{\begingroup
  \output={\global\setbox\partialpage=\vbox{\unvbox255\bigskip}}\eject	% il met toute la page (box255) dans sa 
  \output={\doublecolumnout} \hsize=#1mm \vsize=true89pc}		%  boite privee, avec possibilite de la
				                                    	%  couper en morceaux (\unvbox)

\def\enddoublecolumns{\output={\balancecolumns}\eject
  \endgroup \pagegoal=\vsize}

\def\doublecolumnout{\splittopskip=\topskip \splitmaxdepth\maxdepth
  \dimen@=44pc \advance\dimen@ by-\ht\partialpage
  \setbox0=\vsplit255 to\dimen@ \setbox2=\vsplit255 to\dimen@
  \onepageout\pagesofar \unvbox255 \penalty\outputpenalty}
\def\pagesofar{\unvbox\partialpage
  \wd0=\hsize \wd2=\hsize \hbox to\pagewidth{\box0\hfil\box2}}
\def\balancecolumns{\setbox0=\vbox{\unvbox255} \dimen@=\ht0
  \advance\dimen@ by\topskip \advance\dimen@ by-\baselineskip
  \divide\dimen@ by2 \splittopskip=\topskip
  {\vbadness=10000 \loop \global\setbox3=\copy0
    \global\setbox1=\vsplit3 to \dimen@
    \ifdim\ht3>\dimen@ \global\advance\dimen@ by1pt \repeat}
  \setbox0=\vbox to\dimen@{\unvbox1} \setbox2=\vbox to\dimen@{\unvbox3}
  \pagesofar}



%
%	ecriture 2 pages par page
%       page 241 lpdt, nouvelle edition
%
\newbox\leftpage \newbox\rightpage \newbox\feuille
\newdimen\separateur \separateur=30mm

\def\makefeuille{\setbox\feuille=\hbox{%
                    \hbox to \hsize{\box\leftpage\hss}%
                     \hskip\separateur
                      \hbox to \hsize{\box\rightpage\hss}}}
\def\currentpage{\vtop{\makeheadline\pagebody\makefootline}
                  \advancepageno } 

\def\twopages
    {\ifodd \pageno
      \global\setbox\rightpage=\currentpage
       \makefeuille
        \shipout\box\feuille
         \else \global\setbox\leftpage=\currentpage \fi
          \ifnum\outputpenalty>-20000 \else\dosupereject\fi }
\def\epreuves{\output={\twopages}}

%
%	ecriture 2 pages identiques par page
%       d'apres page 241 lpdt, nouvelle edition
%
\newbox\leftpage \newbox\rightpage \newbox\feuille
\newdimen\separateur \separateur=35.0mm

\def\makefeuille{\setbox\feuille=\hbox{%
                    \hbox to \hsize{\box\leftpage\hss}%
                     \hskip\separateur
                      \hbox to \hsize{\box\rightpage\hss}}}
\def\currentpage{\setbox150=\vtop{\makeheadline\pagebody\makefootline}
                  \advancepageno } 
\def\twopages
      {\currentpage
      \global\setbox\rightpage=\copy150
      \global\setbox\leftpage=\box150
      \makefeuille
      \shipout\box\feuille
      \ifnum\outputpenalty>-20000 \else\dosupereject\fi }
\def\epreuvesacinq{\output={\twopages}}

\catcode`\@=12
