%
% macros d'insertions de figures
%

%
% macro de numerotation	des figures
%

%% definition du compteur de figures			
%\newcount\numfig \numfig=0								% avec_numerotation
%\def\fig{\global\advance\numfig by 1{\bfuit figure~\the\numfig :} \rmuit}		% avec_numerotation
%%\def\fig{fig }										% sans_numerotation
% 
%% macros de referencage des figures			
%
%% referencage manuel					
%%\def\fignum#1{(fig~#1)}								% sans_numerotation
%
%% referencage automatique								% avec_numerotation
%% (attention a ne pas introduire de blancs parasites)					% avec_numerotation
%% Une figure en haut/en bas								% avec_numerotation
%\def\figh{                     (fig \the\numfig)}				% avec_numerotation
%\def\figb{\advance\numfig by 1	(fig~\the\numfig)\advance\numfig by -1}	% avec_numerotation
%% Deux fig en haut/bas et on veut celle de gauche/droite				% avec_numerotation
%\def\dfighg{\advance\numfig by -1(fig~\the\numfig)\advance\numfig by 1}	% avec_numerotation
%\def\dfighd{                     (fig~\the\numfig)}			% avec_numerotation
%\def\dfigbg{\advance\numfig by 1 (fig~\the\numfig)\advance\numfig by -1}	% avec_numerotation
%\def\dfigbd{\advance\numfig by 2 (fig~\the\numfig)\advance\numfig by -2}	% avec_numerotation
%% Deux fig en haut/bas et on veut les deux						% avec_numerotation
%\def\dfigb{\advance\numfig by 1(fig~\the\numfig%				% avec_numerotation
%  \advance\numfig by 1\&~\the\numfig)\advance\numfig by -2}		% avec_numerotation
%\def\dfigh{\advance\numfig by -1(fig~\the\numfig%				% avec_numerotation
%  \advance\numfig by 1\&~\the\numfig)}					% avec_numerotation
%% Trois fig en haut/bas  et on veut celle de gauche/milieu/droite			% avec_numerotation
%\def\tfighg{\advance\numfig by -2(fig~\the\numfig)\advance\numfig by 2}	% avec_numerotation
%\def\tfighm{\advance\numfig by -1(fig~\the\numfig)\advance\numfig by 1}	% avec_numerotation
%\def\tfighd{                     (fig~\the\numfig)}			% avec_numerotation
%\def\tfigbg{\advance\numfig by 1 (fig~\the\numfig)\advance\numfig by -1}	% avec_numerotation
%\def\tfigbm{\advance\numfig by 2 (fig~\the\numfig)\advance\numfig by -2}	% avec_numerotation
%\def\tfigbd{\advance\numfig by 3 (fig~\the\numfig)\advance\numfig by -3}	% avec_numerotation
%% Trois fig en haut/bas  et on veut les trois						% avec_numerotation
%\def\tfigb{\advance\numfig by 1(fig~\the\numfig,%				% avec_numerotation
%  \advance\numfig by 1~\the\numfig~\&% 					% avec_numerotation
%  \advance\numfig by 1~\the\numfig)\advance\numfig by -3}			% avec_numerotation
%\def\tfigh{\advance\numfig by -2(fig~\the\numfig,%			% avec_numerotation
%  \advance\numfig by 1~\the\numfig~\&%					% avec_numerotation
%  \advance\numfig by 1~\the\numfig)}					% avec_numerotation

\def\figp#1{(figure~\ref{#1})}
\def\fig#1{figure~\ref{#1}}

%-----------------------------------------------------------
%-----------------------------------------------------------
%-----------------------------------------------------------

%
% procedures pour inclure des fichiers postscript encapsules 
%     (les unites sont a preciser avec les dimensions)
%

\def\sfigx#1#2{\hbox{\epsfxsize=#1\epsffile{#2.eps}}}
\def\sfigy#1#2{\hbox{\epsfysize=#1\epsffile{#2.eps}}}
\def\sfigxy#1#2#3{\hbox{\epsfxsize=#1\epsfysize=#2\epsffile{#3.eps}}}



%-----------------------------------------------------------
%
% figx : insertion d'un fichier redimensionne en x
%
% #1 = dimension suivant x du dessin
% #2 = nom du fichier a inserer sans l'extension ".eps"
% #3 = legende
% #4 = position de la legende
% #5 = decalage vertical avant
% #6 = decalage vertical apres
%

\def\figx#1#2#3#4#5#6#7{
%  \vskip#4  	 			% decalage
%  \vbox{                             	% ouverture d'une boite V
%    {\ \hfill\sfigx{#1}{#2}\hfill}	% centrage de la figure
%    \vskip#5   	 			% decalage
%    \centerline{\fig #3}             	% centrage d'une legende	
%  }                             	% fermeture d'une boite V
%  \vskip#6				% decalage
 \begin{figure}
  \begin{center}
   \includegraphics[width=#1]{#2.eps}
  \end{center}
  \caption{#3}
  \label{#7}
 \end{figure}
}        		                         

%-----------------------------------------------------------
%
% figxy : insertion d'un fichier redimensionne en x et y
%
% #1 = dimension suivant x du dessin
% #2 = dimension suivant y du dessin
% #3 = nom du fichier a inserer sans l'extension ".eps"
% #4 = legende
% #5 = position de la legende
% #6 = decalage vertical avant
% #7 = decalage vertical apres
%

\def\figxy#1#2#3#4#5#6#7{
  \vskip#5  	 			% decalage
  \vbox{                             	% ouverture d'une boite V
    {\ \hfill\sfigxy{#1}{#2}{#3}\hfill}       % centrage de la figure
    \vskip#6   	 			% decalage
%    \centerline{\fig #4}             	% centrage d'une legende	
  }                             	% fermeture d'une boite V
  \vskip#7				% decalage
}        		                        

%-----------------------------------------------------------
%
% dfigx : insertion de deux figures de meme taille en x 
%         (les unites sont a preciser aves les dimensions)
%
% #1 = dimension suivant x des deux figures      
% #2 = nom du fichier de gauche sans l'extension ".eps"
% #3 = nom du fichier de droite sans l'extension ".eps"
% #4 = legende de gauche 
% #5 = legende de droite 
% #6 = position verticale des legendes
% #7 = decalage vertical avant
% #8 = decalage vertical apres
%

\newdimen\largeur 							

\def\dfigx#1#2#3#4#5#6#7#8{
  \largeur=\textwidth \divide\largeur by2 \advance\largeur by-1truept 	
  \setbox101=\sfigx{#1}{#2}
  \setbox102=\sfigx{#1}{#3}
%  \setbox104=\hbox to \largeur{\hfill\hbox{\fig #4}\hfill}	
%  \setbox105=\hbox to \largeur{\hfill\hbox{\fig #5}\hfill}	
  \vbox{
    \vskip #7
    \hbox to \textwidth{\hfill\box101\hfill\box102\hfill}
    \vskip #6
%    \hbox to \textwidth{\hfill\box104\hfill\box105\hfill}
    \vskip #8
    }
}

%-----------------------------------------------------------
%
% dfigxiso : insertion de deux figures en ligne de dimensions differentes
%            chacune centree dans sa moitie de la page
%         (les unites sont a preciser aves les dimensions)
%
% #1 = dimension suivant x de la figure de gauche
% #2 = dimension suivant x de la figure de droite   
% #3 = nom du fichier de gauche sans l'extension ".eps"
% #4 = nom du fichier de droite sans l'extension ".eps"
% #5 = legende de gauche 
% #6 = legende de droite 
% #7 = position verticale des legendes
% #8 = decalage vertical avant
% #9 = decalage vertical apres
%
\def\dfigxiso#1#2#3#4#5#6#7#8#9{
  \largeur=\textwidth \divide\largeur by2 \advance\largeur by-1truept 	
  \setbox101=\sfigx{#1}{#3}
  \setbox102=\sfigx{#2}{#4}
  \setbox101=\hbox to \largeur{\hfill\box101\hfill}
  \setbox102=\hbox to \largeur{\hfill\box102\hfill}
%  \setbox104=\hbox to \largeur{\hfill\hbox{\fig #5}\hfill}	
%  \setbox105=\hbox to \largeur{\hfill\hbox{\fig #6}\hfill}	
  \vbox{
    \vskip #8
    \hbox to \textwidth{\hfill\box101\hfill\box102\hfill}
    \vskip #7
    \hbox to \textwidth{\hfill\box104\hfill\box105\hfill}
    \vskip #9
    }
}

%-----------------------------------------------------------
%
% dfigxdif : insertion de deux figures en ligne de dimensions differentes
%            avec les blancs repartis egalement entre les figures
%         (les unites sont a preciser aves les dimensions)
%
% #1 = dimension suivant x de la figure de gauche
% #2 = dimension suivant x de la figure de droite   
% #3 = nom du fichier de gauche sans l'extension ".eps"
% #4 = nom du fichier de droite sans l'extension ".eps"
% #5 = legende de gauche 
% #6 = legende de droite 
% #7 = position verticale des legendes
% #8 = decalage vertical avant
% #9 = decalage vertical apres
%

\def\dfigxdif#1#2#3#4#5#6#7#8#9{
  \setbox101=\sfigx{#1}{#3}
  \setbox102=\sfigx{#2}{#4}
%  \setbox104=\hbox to #1{\hfill\hbox{\fig #5}\hfill}	
%  \setbox105=\hbox to #2{\hfill\hbox{\fig #6}\hfill}	
  \vbox{
    \vskip #8
    \hbox to \textwidth{\hfill\box101\hfill\box102\hfill}
    \vskip #7
%    \hbox to \textwidth{\hfill\box104\hfill\box105\hfill}
    \vskip #9
    }
}

%-----------------------------------------------------------
%
% tfigx : insertion de trois figures de meme dimension 
%         (les unites sont a preciser aves les dimensions)
%
% #1 = dimension suivant x de la figure de gauche
% #2 = nom du fichier de gauche sans l'extension ".eps"
% #3 = nom du fichier du centre sans l'extension ".eps"
% #4 = nom du fichier de droite sans l'extension ".eps"
% #5 = legendes de l'image de gauche
% #6 = legendes de l'image du milieu
% #7 = legendes de l'image de droite
% #8 = decalage vertical avant
% #9 = decalage vertical apres
%

\def\tfigx#1#2#3#4#5#6#7#8#9{
  \largeur=\textwidth \divide\largeur by3 \advance\largeur by-1truept 	
  \setbox101=\sfigx{#1}{#2}
  \setbox102=\sfigx{#1}{#3}
  \setbox103=\sfigx{#1}{#4}
  \setbox101=\hbox to \largeur{\hfill\box101\hfill}
  \setbox102=\hbox to \largeur{\hfill\box102\hfill}
  \setbox103=\hbox to \largeur{\hfill\box103\hfill}
%  \setbox104=\hbox to \largeur{\hfill\hbox{\fig #5}\hfill}	
%  \setbox105=\hbox to \largeur{\hfill\hbox{\fig #6}\hfill}	
%  \setbox106=\hbox to \largeur{\hfill\hbox{\fig #7}\hfill}	
  \vbox{
    \vskip #8
     \hbox to \textwidth{\hfill\box101\hfill\box102\hfill\box103\hfill}
     \vskip 2mm
%     \hbox to \textwidth{\hfill\box104\hfill\box105\hfill\box106\hfill}
     \vskip #9
    }
}

%-----------------------------------------------------------
%
% tfigxiso : insertion de trois figures en ligne de dimension et de legendes differentes 
%            chacune centree dans sa moitie de la page
%         (les unites sont a preciser aves les dimensions)
%
% #1 = dimension suivant x de la figure de gauche
% #2 = dimension suivant x de la figure du milieu
% #3 = dimension suivant x de la figure de droite
% #4 = nom du fichier de gauche sans l'extension ".eps"
% #5 = nom du fichier du centre sans l'extension ".eps"
% #6 = nom du fichier de droite sans l'extension ".eps"
% #7 = legendes de l'image de gauche 
% #8 = legendes de l'image du milieu
% #9 = legendes de l'image de droite
%

\def\tfigxiso#1#2#3#4#5#6#7#8#9{
  \largeur=\textwidth \divide\largeur by3 \advance\largeur by-1truept 	
  \setbox101=\sfigx{#1}{#4}
  \setbox102=\sfigx{#2}{#5}
  \setbox103=\sfigx{#3}{#6}
  \setbox101=\hbox to \largeur{\hfill\box101\hfill}
  \setbox102=\hbox to \largeur{\hfill\box102\hfill}
  \setbox103=\hbox to \largeur{\hfill\box103\hfill}
%  \setbox104=\hbox to \largeur{\hfill\hbox{\fig #7}\hfill}	
%  \setbox105=\hbox to \largeur{\hfill\hbox{\fig #8}\hfill}	
%  \setbox106=\hbox to \largeur{\hfill\hbox{\fig #9}\hfill}	
  \vbox{
    \vskip 2mm
     \hbox to \textwidth{\hfill\box101\hfill\box102\hfill\box103\hfill}
     \vskip 2mm
%     \hbox to \textwidth{\hfill\box104\hfill\box105\hfill\box106\hfill}
     \vskip 0mm
    }
}

%-----------------------------------------------------------
%
% tfigxiso : insertion de trois figures en ligne de dimension et de legendes differentes 
%            avec les blancs repartis egalement entre les figures
%         (les unites sont a preciser aves les dimensions)
%
% #1 = dimension suivant x de la figure de gauche
% #2 = dimension suivant x de la figure du milieu
% #3 = dimension suivant x de la figure de droite
% #4 = nom du fichier de gauche sans l'extension ".eps"
% #5 = nom du fichier du centre sans l'extension ".eps"
% #6 = nom du fichier de droite sans l'extension ".eps"
% #7 = legendes de l'image de gauche 
% #8 = legendes de l'image du milieu
% #9 = legendes de l'image de droite
%

\def\tfigxdif#1#2#3#4#5#6#7#8#9{
  \setbox101=\sfigx{#1}{#4}
  \setbox102=\sfigx{#2}{#5}
  \setbox103=\sfigx{#3}{#6}
  \setbox104=\hbox to #1{\hfill\hbox{\fig #7}\hfill}	
  \setbox105=\hbox to #2{\hfill\hbox{\fig #8}\hfill}	
  \setbox106=\hbox to #3{\hfill\hbox{\fig #9}\hfill}	
  \vbox{
    \vskip 2mm
     \hbox to \textwidth{\hfill\box101\hfill\box102\hfill\box103\hfill}
     \vskip 2mm
     \hbox to \textwidth{\hfill\box104\hfill\box105\hfill\box106\hfill}
     \vskip 2mm
    }
}


%-----------------------------------------------------------
%-----------------------------------------------------------
%-----------------------------------------------------------

%
% macros d'insertion de tableaux
%


 % definition du compteur de tableaux			% avec_numerotation
%\newcount\numtab \numtab=0				% avec_numerotation
%
%% % macro de numerotation					% numerotation
%%\def\tabn#1{tab~#1 :}						% sans_numerotation
%\def\tabn{\global\advance\numtab by 1 				% avec_numerotation
%%	\rmuit							% avec_numerotation
%	TABLEAU~\the\numtab : }			% avec_numerotation
% 
%% % macros de referencage des tableaux				% numerotation
%
%% % referencage manuel						% sans_numerotation
%%\def\tac#1{(tac~#1)}						% sans_numerotation
%
% % referencage automatique					% avec_numerotation
% % le tableau est en haut					% avec_numerotation
%\def\tabh{tableau~\the\numtab}			% avec_numerotation
% % le tableau est deux tableaux avant				% avec_numerotation
%\def\tabhh{{\advance\numtab by -1%				% avec_numerotation
%tableau~\the\numtab}}		% avec_numerotation
% % le tableau est en bas					% avec_numerotation
%\def\tabb{{\advance\numtab by 1%				% avec_numerotation
%tableau~\the\numtab}}		% avec_numerotation
% % le tableau est deux tableaux en bas				% avec_numerotation
%\def\tabbb{{\advance\numtab by 2%				% avec_numerotation
%tableau~\the\numtab}}		% avec_numerotation

\def\tabp#1{(tableau~\ref{#1})}
\def\tab#1{tableau~\ref{#1}}

%-----------------------------------------------------------

%
% insertion de tableau
% les unites sont a preciser avec les dimensions
% 
% #1 le titre
% #2 le tableau
% #3 commentaire eventuel
% #4 ecart entre le titre et le tableau
% #5 ecart entre le tableau et le commentaire
% #6 ecart avant le tableau 
% #7 ecart apres le tableau
%

\def\tableau#1#2#3#4#5#6#7#8{
   \begin{table}
 	\vbox{
%		\setbox100=\ctitre{#1}		% sans_numerotation
%		\setbox100=\ctitre{\caption{#1}\label{#8}}	% avec_numerotation
		\setbox101=\vbox{#2}
		\setbox102=\ctitre{#3}

		\vskip #6 
%		\centerline{\box100}
                \caption{#1}\label{#8}
		\vglue #4 plus 2pt minus 2pt 
		\centerline{\box101}
		\vglue #5 plus 2pt minus 2pt 
		\centerline{\box102}
		\vskip #7
	}
   \end{table}
}




