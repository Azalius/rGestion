% DEFINITION DES GLOSSAIRES

%%%%%%%%%%%%%%%%%%%%%%%%%%%%%
%		ACRONYMES
%%%%%%%%%%%%%%%%%%%%%%%%%%%%%

\newacronym{tcd}{TCD}{Transition Choc Détonation}
\newacronym{tdd}{TDD}{Transition Déflagration Détonation}
\newacronym{hmx}{HMX}{High Melting point eXplosives}
\newacronym{rdx}{RDX}{Research Development compound eXplosive}
\newacronym{tnt}{TNT}{TriNitroToluène - Tolite}
\newacronym{onta}{ONTA}{Oxynitrotriazole}
\newacronym{cea}{CEA}{Commissariat à l'Energie Atomique et aux Energies Alternatives}
\newacronym{dga}{DGA}{Direction Générale de l'Armement}
\newacronym{ensta}{ENSTA}{Ecole Nationale Supérieure des Techniques Avancées}
\newacronym{siame}{SIAME}{SImulation Aéro-thermo-chimique de la Mécanique des Explosifs}
\newacronym{mris}{MRIS}{Mission pour la Recherche et l'Innovation Scientifique}
\newacronym{edo}{EDO}{équation différentielle ordinaire}
\newacronym{svd}{SVD}{Single Value Decomposition}
\newacronym{eos}{EOS}{\textit{Equation Of State}}
\newacronym{ZND}{ZND}{Zeldovitch - Von Neumann - Döring}
\newacronym{gop}{GOP}{Générateur d'Onde Plane}
\newacronym{cj}{CJ}{Chapman-Jouguet}
\newacronym{murat}{MURAT}{MUnition à Risques ATténués}
\newacronym{etr}{ETR}{Equation du Transfert Radiatif}
\newacronym{jmak}{JMAK}{Johnson-Mehl-Avrami-Kolmogorov}
\newacronym{vh}{VH}{Vélocimétrie Hétérodyne}
\newacronym{idf}{IDF}{Interférométrie Doppler Fibrée}
\newacronym{idl}{IDL}{Interférométrie Doppler Laser}
\newacronym{rif}{RIF}{Radio InterFéromètre}
\newacronym{jwl}{JWL}{Jones - Wilkins - Lee}
\newacronym{iad}{IAD}{Indice d'Aptitude à la Détonation}

%%%%%%%%%%%%%%%%%%%%%%%%%%%%%
%		GLOSSAIRE
%%%%%%%%%%%%%%%%%%%%%%%%%%%%%


\newglossaryentry{detonique}{name=détonique,description={Discipline concernant l'étude de la détonation et des autres modes de décomposition des matériaux énergétiques, de leurs causes et de leurs effets sur les matériaux connexes \cite{dico_pyro}}}

\newglossaryentry{detonation}{name=détonation,description={Réaction de décomposition exothermique et auto-entretenue d'une substance explosible par onde de choc, dont la vitesse de propagation, de l'ordre de plusieurs milliers de \si{\metre\per\second}, est supérieure à la célérité du son dans la matière \cite{dico_pyro}}}

\newglossaryentry{deflagration}{name=déflagration,description={Réaction de décomposition exothermique auto-entretenue d'une substance explosible dont la vitesse apparente est inférieure à la vitesse du son dans la matière et supérieure à la vitesse du son dans l'air [...] \cite{dico_pyro}}}

\newglossaryentry{combustion}{name=combustion,description={Réaction de décomposition exothermique auto-entretenue d'un matériau énergé\-tique ou d'un mélange oxydant-réducteur, dont le front de réaction se propage à une vitesse inférieure à la vitesse du son dans l'air, quelques \si{\milli\metre\per\second} à quelques \si{\metre\per\second} \cite{dico_pyro}}}


\newglossaryentry{amorcage}{name=amorçage,description={Phénomène donnant naissance à une détonation \cite{dico_pyro}}}

\newglossaryentry{explosif}{name=explosif,description={Matière ou substance \gls{explosible} utilisée pour les effets de sa détonation comme charge primaire, charge renforçatrice ou charge principale \cite{dico_pyro}}}

\newglossaryentry{explosible}{name=explosible,description={Substance qui, par réaction chimique, dégage des gaz ou des flux thermiques dans des conditions telles qu'il en résulte des dommages aux alentours \cite{dico_pyro}}}

\newglossaryentry{compositionsexplosives}{name=compositions explosives,description={Mélange à base de substances explosives dont le régime de décomposition fonctionnel est la détonation. Par simplification, on appelle souvent ces substances : explosifs \cite{dico_pyro}}}

\newglossaryentry{pyrotechnie}{name=pyrotechnie,description={Ensemble des connaissances scientifiques et des moyens mis en œuvre dans le domaine des substances explosibles \cite{dico_pyro}}}

\newglossaryentry{materiauenergetique}{name=matériau énergétique,description={Matériau dont la décomposition s'accompagne d'un dégagement d'énergie \cite{dico_pyro}}}

\newglossaryentry{detonateur}{name=détonateur,description={Initiateur dont la fonction consiste à transformer directement une énergie extérieure (méca\-nique, électrique, thermique, etc.) en une onde de choc suffisante pour amorcer un explosif secondaire. \cite{dico_pyro}}}

\newglossaryentry{muratisation}{name=muratisation,description={Action de muratiser. Action de concevoir ou améliorer une munition pour la rendre moins sensible lors d'une agression accidentelle. \cite{dico_pyro}}}

\newglossaryentry{ontalite}{name=ontalite,description={Désigne une composition énergétique à base de TNT et d'ONTA}}
\newglossaryentry{octolite}{name=octolite,description={Désigne une composition énergétique à base de TNT et de HMX.}}
\newglossaryentry{propergol}{name=propergol,description={Produit comprenant un ou des ergols, soit séparés, soit réunis, pour former un mélange ou composé apte à fournir l'énergie de propulsion d'un moteur-fusée}}

\newglossaryentry{initiation}{name=initiation,description={En pyrotechnie, stimulus (d'origine thermique, mécanique ou autre) donnant naissance à la réaction d'un matériau énergétique}}